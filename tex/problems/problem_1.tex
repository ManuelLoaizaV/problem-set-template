
\begin{statement}{1}
    Este es el enunciado de un problema.
    Colocar el enunciado de un problema de un color diferente
    permite distinguir entre el problema y la soluci\'on.
\end{statement}

\begin{proof}
    Tipea tus soluciones en esta secci\'on.
    Usa las definiciones, lemas y ejemplos cuando sean necesarios.
    \begin{defn}
        Definimos \(\exp(x)\) para \(x \in \BR\) el valor de \[\sum_{i = 0}^\infty\frac{x^i}{i!}\].
    \end{defn}
    Al igual que en la definici\'on anterior,
    usar ecuaciones en l\'ineas separadas cuando sea posible,
    esto hace tu problema m\'as le\'ible. Usa \texttt{align*} para lista de igualdades:
    \begin{align*}
        0 &= 0 + 0 + 0 + 0 + \dots\\
        &= (1 - 1) + (1 - 1) + \dots \\
        &= 1 + (-1 + 1) + (-1 + 1) + \dots \\
        &= 1 + 0 + 0 + 0 \dots \\
        &= 1.
    \end{align*}
    Si necesitas listar cosas, usa \texttt{enumerate} o \texttt{itemize}. Por ejemplo:
    \begin{enumerate}
        \item Debo terminar la lista de ejercicios de \'Algebra Multilineal.
        \item Debo terminar la plantilla para Teor\'ia de Juegos.
        \item Debo volver a entrenar Programaci\'on Competitiva.
    \end{enumerate}
    Todas las secciones \texttt{proof} terminan con el cuadrado que simboliza el hecho de que ha concluido la demostraci\'on.
\end{proof}